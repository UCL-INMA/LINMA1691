\section{Coloriages de sommets}
\subsection{Coloriages de sommets}
\index{coloriage!coloriage de noeuds}
\index{coloriage!coloriage de noeuds propre}
\index{coloriage!coloriage minimum}
\begin{mydef}
  Un \emph{coloriage de noeuds} est l’attribution à chaque noeud d’une couleur.

  Le coloriage est dit \emph{propre} si des noeuds adjacents ont des couleurs différentes.

  Un coloriage \emph{minimum} est un coloriage propre qui emploie un nombre minimal de couleurs.
\end{mydef}

\subsection{Nombre chromatique, cliques et degrés}
\index{chromatique!nombre chromatique}
\begin{mydef}
  Le \emph{nombre chromatique}, noté $\chi$, d’un graphe est le nombre minimal de couleurs d’un coloriage propre de sommets du graphe.
\end{mydef}

\begin{mytheo}
  $|$clique max$| \leq \chi$
  \begin{proof}
    Dans une clique, comme tout le monde est adjacent à tout le monde, tous les noeuds doivent être différents.
    Donc il faut au moins une couleur différentes à chaque membre de la plus grande de clique ce qui donne
    l'inégalité à démontrer.
  \end{proof}
\end{mytheo}

\begin{mytheo}
  $\chi \leq \degmax + 1$
  \begin{proof}
    Par induction sur le nombre de noeuds $n$.
    \begin{description}
      \item[Initialisation]
        Trivial pour $n = 1$.
      \item[Induction]
        Soit $G$ graphe de degré max $d$ de $n$ noeuds. On enlève un noeuds $x$. $G-\{x\}$ est un graphe de $n-1$ noeuds de degré max $\leq d$.

        Par hypothèse d'induction, il existe un coloriage propre de $G-\{x\}$.
        On rétablis $x$ et ses arêtes. $x$ a au plus $d $ voisins.

        \begin{itemize}
          \item Si  $G-\{x\}$ a $\leq d$ couleurs, alors ont peut utiliser une $d$ème couleur pour $x$.
          \item  Si $G-\{x\}$ a $\leq d+1$ couleurs, alors $x$ a au plus $d$ voisins. Il existe donc une couleur non-utilisée, libre pour $x$.
        \end{itemize}
    \end{description}
  \end{proof}
\end{mytheo}
\subsection{Polynôme chromatique}
Pour un graphe $G$, le nombre de coloriage propres à $k$ couleurs est noté $\pi_k(G)$.
\begin{mytheo}
  Pour toute arête $e$ du graphe $G$, $\pi_k(G) = \pi_k(G-e) - \pi_k(G.e)$ où $G.e$ est le graphe obtenu en contractant l'arête $e$.
  \begin{proof}
  Soit $u$ et $v$ les deux extrémités de $e$.
	\begin{itemize}
	\item $\pi_k(G-e)$ est le nombre de coloriages des noeuds de $G$ à moins de $k$ couleurs qui sont propres sauf possiblement entre $u$ et $v$. On peut avoir $u$ et $v$ de même couleur.
	\item $\pi_k(G.e)$ est le nombre de coloriages des noeuds de $G$ à moins de $k$ couleurs qui sont propres sauf entre $u$ et $v$. On doit avoir $u$ et $v$ de même couleur.
	\end{itemize}
	On a donc finalement $\pi_k(G-e) - \pi_k(G.e)$ est le nombre de coloriages des noeuds de $G$ à moins de $k$ couleurs qui sont propres même entre $u$ et $v$ : on doit avoir $u$ et $v$ de couleurs différentes.
  \end{proof}
\end{mytheo}

\begin{mycorr} [Birkhoff]
  Pour un graphe $G$ à $n$ noeuds, $\pi_k (G)$ est un polynôme monique de degré $n$, de terme constant nul et dont les coefficients alternent en signe.
  \begin{proof}
    Par récurrence en utilisant $\pi_k(G) = \pi_k(G-e)-\pi_k(G.e)$ sur le nombre d'arêtes. Le cas de base correspond à un noeud isolé.
  \end{proof}
\end{mycorr}
