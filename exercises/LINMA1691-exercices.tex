\documentclass[11pt,a4paper]{article}

% French
\usepackage[utf8x]{inputenc}
\usepackage[frenchb]{babel}
\usepackage[T1]{fontenc}
\usepackage{lmodern}
\usepackage{url}

% Math symbols
\usepackage{amsmath}
\usepackage{amssymb}
\usepackage{amsthm}
\usepackage{subfigure} %Allows to have several figures on the same line.
\usepackage{hyperref} %Allows to make references (\ref{}), pdf links are now clickable

% Theorem and definitions
\theoremstyle{definition}
\newtheorem{mydef}{Définition}
\newtheorem{mynota}[mydef]{Notation}
\newtheorem{myprop}[mydef]{Propriétés}
\newtheorem{myrem}[mydef]{Remarque}
\newtheorem{myform}[mydef]{Formules}
\newtheorem{mycorr}[mydef]{Corrolaire}
\newtheorem{mytheo}[mydef]{Théorème}
\newtheorem{mylem}[mydef]{Lemme}
\newtheorem{myexem}[mydef]{Exemple}
\newtheorem{myalgo}[mydef]{Algorithme}

\newcommand{\bigoh}{\mathcal{O}}

\usepackage{tikz}

\definecolor{mygreen}{rgb}{0,0.6,0}
\definecolor{mygray}{rgb}{0.5,0.5,0.5}
\definecolor{mymauve}{rgb}{0.58,0,0.82}

\tikzstyle{vertex}=[circle,fill=gray!50,minimum size=15pt,inner sep=0pt]
\tikzstyle{visited}=[circle,fill=green!25,minimum size=15pt,inner sep=0pt]
\tikzstyle{unvisited}=[circle,fill=blue!25,minimum size=15pt,inner sep=0pt]

\newcommand{\W}{\ {\color{red} \textbf{!!}} \ }


\usepackage{verbatim} % for comment

\usepackage[framemethod=tikz]{mdframed}
%\usepackage{tikzpagenodes}
\usepackage{hyperref} %Allows to make references (\ref{}), pdf links are now clickable.
\usetikzlibrary{calc, arrows}

% greyarrow has much better handling of page breaks, see the link on stackoverflow for more info
\def\SolStyle{greyarrow}
\ifthenelse{\isundefined{\SolStyle}}{\def\SolStyle{greyarrow}}{}
\ifthenelse{\equal{\SolStyle}{greyarrow}}
{
  \input{greyarrow.tex}
}
{
  \input{boxes.tex}
}


\ifthenelse{\isundefined{\Sol}}{\def\Sol{true}}{}
\ifthenelse{\equal{\Sol}{false}}
{
  \newenvironment{solution}{\expandafter\comment}{\expandafter\endcomment}
}
{
  \newmdenv[style=mysquare]{solution}
}

\newcommand{\nosolution}
{Cet exercice ne contient pas encore de solution.
Vous êtes invité à nous en soumettre une à l'adresse suivante
\begin{center}
\url{https://github.com/blegat/LINMA1691}
\end{center}
ou par mail.}

\parindent0mm

\setlength{\textwidth}{15cm}
\setlength{\oddsidemargin}{0.50cm}
\setlength{\textheight}{8.7in}
\setlength{\topmargin}{-0.5in}

\newcommand{\R}{{\mbox{\bf R}}}

\title{INMA1691 - Théorie et Algorithmique des Graphes : Exercices}
\author{
  Robin Ballarini
  \and Annelies Bauwens
  \and Gilles Bertrand
  \and Armand Bosquillon de Jenlis
  \and Romain Capron
  \and Arnaud Cerckel
  \and Charlotte Cirriez
  \and Simon Claessens
  \and Gaëtan Collart
  \and Matthieu Constant
  \and Xavier Crochet
  \and Gatien De Callataÿ
  \and François Dederichs
  \and Sébastien De Fauw
  \and François Delcourt
  \and Arnaud de Lhoneux
  \and Martin De Neuville
  \and Guillaume Derval
  \and Alexandre de Touzalin
  \and Gauthier Feuillen
  \and Florentin Goyens
  \and Arnaud Jacques
  \and Antoine Hilhorst
  \and Benoît Legat
  \and Arthur Losseau
  \and Alexis Pierret
  \and Thérèse Plissart
  \and François Raucent
  \and Félicien Schiltz
  \and Mélanie Sedda
  \and Benoît Sluysmans
  \and Nicolas Stevens
  \and Harold Taeter
  \and Kim Van Den Eeckhaut
  \and Geoffroy Vanderreydt
  \and Antoine Van Malleghem
  \and Nicolas Vico}


\begin{document}
%%%%%%%%%%%%%%%%%%%%%%%%%%%%%%%%%%%%%%%%%%%%%%%%%%%%%%%%%%%%%%%
%%%%%%%%%%%%%%%%%%%%%%%% PAGE DE GARDE %%%%%%%%%%%%%%%%%%%%%%%%
%%%%%%%%%%%%%%%%%%%%%%%%%%%%%%%%%%%%%%%%%%%%%%%%%%%%%%%%%%%%%%%
\pagenumbering{gobble}% Remove page numbers
\maketitle
\begin{center}
  \textit{Basé sur les séances de TP données par
  Adeline Decuyper et Romain Hollanders}\\
  \includegraphics[width=100pt]{../img/logo}\\
  \textbf{Code source et bug tracker}\\
  \url{https://github.com/blegat/LINMA1691}
\end{center}
\newpage
\clearpage

%%%%%%%%%%%%%%%%%%%%%%%%%%%%%%%%%%%%%%%%%%%%%%%%%%%%%%%%%%%%%%%
%%%%%%%%%%%%%%%%%%%%%%% Table Of Contents %%%%%%%%%%%%%%%%%%%%%
%%%%%%%%%%%%%%%%%%%%%%%%%%%%%%%%%%%%%%%%%%%%%%%%%%%%%%%%%%%%%%%
\pagenumbering{arabic}% Arabic page numbers (and reset to 1)
\tableofcontents

\section{Graphes connexes, eulériens et bipartis}

\section{Les plus courts chemins}
\index{graphe!graphe pondéré}
\begin{mydef}
  Une \emph{fonction de poids} sur un graphe ($V$, $E$, $\varphi$) est une fonction $E \to \mathbb{R}$. Un \emph{graphe pondéré} est un graphe muni d’une fonction de poids. Le \emph{poids} ou la \emph{longueur} d’un parcours est la somme des poids des arêtes qui le compose.
\end{mydef}

\begin{mytheo} [Plus court chemin et plus court parcours]
  Pour un graphe avec une fonction de poids $\geq 0$, si le plus court parcours entre $u$ et $v$ est de longueur $d$, alors le plus court chemin entre $u$ et $v$ est aussi de longueur $d$.
  \begin{proof}
     \href{https://dl.dropboxusercontent.com/u/44092863/Graph_Theory_Romain_Capron.pdf}{Voir notes} \textcolor{red}{TODO}
  \end{proof}
\end{mytheo}

\subsection{Algorithme de Dijkstra}
\index{algorithme!algorithme de Dijkstra}
\begin{myalgo}[Algorithme de Dijkstra]
\end{myalgo}

\begin{myexem}
  On cherche les distances à partir de $a$.\\
	\begin{center}
	 \begin{tikzpicture}
    \node[vertex] at (0, 2) (a) {\tiny $A$};
    \node[vertex] at (2, 1) (b) {\tiny $B$};
    \node[vertex] at (1, -2) (c) {\tiny $C$};
    \node[vertex] at (-1, -2) (d) {\tiny $D$};
    \node[vertex] at (-2, 1) (e) {\tiny $E$};
    \draw[->] (a) edge node[anchor = south] {\tiny $50$} (b);
		 \draw[->] (c) edge node[anchor = north] {\tiny $5$} (b);
		\draw[->] (c) edge node[anchor = north] {\tiny $50$} (d);
		\draw[->] (e) edge node[anchor = south] {\tiny $10$} (d);
		\draw[->] (a) edge node[anchor = south] {\tiny $10$} (e);
		\draw[->] (d) edge node[anchor = north] {\tiny $5$} (a);
		\draw[->] (a) edge node[anchor = south] {\tiny $30$} (c);
		\draw[->] (d) edge node[anchor = north] {\tiny $20$} (b);
		 \end{tikzpicture}
\end{center}
\begin{center}
\begin{tabular}{|c|c|ccccc|}
\hline
$u'$ & $S$ & $l(a)$ & $l(b)$ & $l(c)$ & $l(d)$ & $l(e)$\\
\hline
$a$ & $\lbrace a \rbrace$ & $\boxed{0}$ & $\infty$ & $\infty$ & $\infty$ &$\infty$\\
$a$ & $\lbrace a,e \rbrace$ && 50 & 30 & $\infty$ & $\boxed{10}$\\
$e$ & $\lbrace a,e,d \rbrace$ && 50 & 30 & $\boxed{20}$ &\\
$d$ & $\lbrace a,e,d,c \rbrace$ && 40 & $\boxed{30}$ & &\\
$c$ & $\lbrace a,e,d,c,b \rbrace$ && $\boxed{35}$ & & &\\
\hline
\end{tabular}
\end{center}
\end{myexem}

\begin{mytheo} [L'algorithme de Dijkstra fonctionne]
  Après chaque MISE A JOUR DE $\ell$ dans l’algorithme, les deux propriétés suivantes sont vérifiées :
  \begin{itemize}
    \item pour $v \in S$, $\ell(v) = d(u_0, v)$ et le chemin le plus court de $u_0$ à $v$ reste dans $S$;
    \item pour $v \notin S$, $\ell(v) \geq d(u_0, v)$, et $\ell(v)$ est la longueur du plus court chemin de $u_0$ vers $v$ dont tous les noeuds internes sont dans $S$.
  \end{itemize}
  \begin{proof}
     \href{https://dl.dropboxusercontent.com/u/44092863/Graph_Theory_Romain_Capron.pdf}{Voir notes} \textcolor{red}{TODO}
  \end{proof}
\end{mytheo}

\begin{mycorr} [L'algorithme de Dijkstra est correct]
  L’algorithme de Dijkstra est correct.
\end{mycorr}

\begin{mytheo} [L’algorithme de Dijkstra est quadratique]
  L’algorithme de Dijkstra sur un graphe se termine en un temps de l’ordre $n^2$ .
  \begin{proof}
    \begin{itemize}
		\item A chaque passage de boucle $\left\lbrace v \in S \right\rbrace $ décroit de 1. Ce qui implique $n$ passages de boucles.
		\item A chaque boucle, "$\forall v \in S$" fait $O(n-|S|)$ opérations; et "trouver $v_{min}$" fait $O(n-|S|)$ opérations également.
		\item Cela implique un temps total $O(n + (n-1) + (n-2) + ... + 1) = O\left( \frac{n(n-1)}{2} \right) = O(n^2)$ puisqu'on considère le temps à une constante près. 
		\end{itemize}
  \end{proof}
\end{mytheo}

\index{graphe!graphe dirigé}
\begin{mydef}
  Un \emph{graphe dirigé} est un triplet ($V$, $E$, $\varphi$), où :\\
  - $V$ est un ensemble dont les éléments sont appelés sommets ou noeuds; \\
  - $E$ est un ensemble dont les éléments sont appelés arêtes; \\
  - $\varphi$ est une fonction, dîte fonction d'incidence, qui associe à chaque arête un couple de sommets. \\
\end{mydef}

\begin{myexem}
  \href{https://dl.dropboxusercontent.com/u/44092863/Graph_Theory_Romain_Capron.pdf}{Voir notes} \textcolor{red}{TODO}
\end{myexem}

\subsection{Semi-anneaux}
\begin{mydef}
Un semi-anneau est un ensemble muni de deux opérations ($\oplus,\otimes$) et ses propriétés.
\end{mydef}
\begin{mydef}
Un anneau est un ensemble muni de trois opérations ($\oplus,\otimes,\ominus$) et ses propriétés.
\end{mydef}
\begin{mydef}
Un corps est un ensemble muni de quatre opérations ($\oplus,\otimes,\ominus,\oslash$) et ses propriétés.
\end{mydef}
Par exemple, une propriété peux être la distributivité : 
$$a\otimes (b \oplus c) = (a \otimes b)\oplus(a \otimes c) $$
On peut par exemple définir des matrices avec un semi-anneau : 
\begin{eqnarray}
A \oplus B &=& (a_{ij} \oplus b_{ij})_{ij}\\
A \otimes B &=& (\sum_k a_{ik} \otimes b_{kj})_{ij}
\end{eqnarray}
Le $\oplus$ définit la somme matricielle et le $\otimes$ le produit matriciel.\\
Pour calculer le plus court chemin, on peut définir l'itération : 
\begin{eqnarray}
M_0 &=& I\\
M_1 &=& I \oplus (M_0 \otimes A)\\
M_2 &=& I \oplus (M_1 \otimes A)\\
 &\vdots &\\
M_{k+1} &=& I \oplus (M_k \otimes A)\\
\end{eqnarray}
On a fini de converger quand $M_k = M_{k+1}$. Dijkstra peut être vu comme une manière efficace d'implémenter cette itération.
% Packages utilisés :
%
% \usepackage[french]{babel} 
% \usepackage[utf8x]{inputenc}
% \usepackage{amsmath}
% \usepackage{amssymb}
% \usepackage{tikz}
% \usepackage{tkz-graph}
%
% Il y a aussi besoin d'un environnement solution

\section{TP3}

\paragraph{1. }
	Lorsque toutes les arêtes d'un graphe sont de longueur $1$, la recherche en largeur dans le graphe permet de trouver le plus court chemin du nœud s au nœud t. Quelle est la complexité de cet algorithme? Quel est le pire cas?\\

\begin{solution}
	Étant donné que cet algorithme doit, au plus, parcourir toutes les arêtes pour trouver le chemin, on a une complexité en $O(\vert E \vert)$, avec $\vert E \vert$ le nombre d'arêtes dans le graphe.\\

	Le pire des cas (celui où le nombre d'arêtes est le plus élevé) est le cas du graphe complet. En effet, par le théorème des poignées de mains, on a que $\vert E \vert = \frac{n(n-1)}{2}$. La complexité est donc $O(n^2)$, avec $n$ le nombre de nœuds dans le graphe.
\end{solution}

\paragraph{7. }
	Vous possédez un billet de $p$ euros et vous souhaitez le changer en pièces de $a_1, a_2, ..., a_k$ euros (tous les montants étant entiers). Est-ce possible? Si oui, avec quel nombre de pièces? Formulez ce problème comme un problème de plus court chemin dans un graphe.\\
	
\begin{solution}
	Créons un digraphe de la manière suivante : $p+1$ nœuds numéroté de $0$ à $p$ et des arêtes telles qu'une arête associée à une pièce de valeur $a_k$ relie un nœud $i$ à un nœud $i+a_k$\footnote{à condition de ces deux nœuds à relier soient bien compris entre $0$ et $p$}. Il suffit ensuite d'appliquer l'algorithme de Dijkstra au graphe créé pour chercher le plus petit chemin entre les nœuds $0$ et $p$. Si l'algorithme ne trouve pas de chemin, il est alors impossible d'échanger le montant $p$ avec les pièces disponibles. Vous pouvez regarder quelques exemples ci-dessous.\\
	
	Premièrement, un exemple où il n'existe pas de parcours du nœud $0$ au nœud $p$ avec $p=3$ et $a_1 = 2$. En effet, il n'est pas possible d'échanger $3$ euros avec des pièces de $2$ euros.\\
	
\begin{figure}[h!]
	\centering
	\begin{tikzpicture}[scale=0.75,transform shape]
		\Vertex[x=-10,y=0]{0}
 		\Vertex[x=-8,y=0]{1}
		\Vertex[x=-6,y=0]{2}
		\Vertex[x=-4,y=0]{3}
		\tikzset{LabelStyle/.style =   {draw}}
	 	\tikzstyle{EdgeStyle}=[bend left]
	 	\Edge(0)(2)
	 	\tikzstyle{EdgeStyle}=[bend right]
		\Edge(1)(3)
 	\end{tikzpicture}
\end{figure}

	Deuxièmement, un exemple ou il existe un parcours de $0$ à $p$ avec $p=5$, $a_1=2$ et $a_2=1$. Le chemin le plus court est tracé en rouge et correspond à $2$ pièces de $2$ et une pièce de $1$. Il existe deux autres chemins de même longueur (donc équivalents).\\
	
\begin{figure}[h!]
	\centering
	\begin{tikzpicture}[scale=0.75,transform shape]
		\Vertex[x=-10,y=0]{0}
		\Vertex[x=-8,y=0]{1}
		\Vertex[x=-6,y=0]{2}
		\Vertex[x=-4,y=0]{3}
		\Vertex[x=-2,y=0]{4}
		\Vertex[x=0,y=0]{5}
		\tikzset{LabelStyle/.style =   {draw}}
		\tikzstyle{EdgeStyle}=[bend left,double=red]
		\Edge(0)(1)  
		\tikzstyle{EdgeStyle}=[bend left]
		\Edge(1)(2)
		\Edge(2)(3)
		\Edge(3)(4)
		\Edge(4)(5)
		\tikzstyle{EdgeStyle}=[bend right, double = red]
		\Edge(1)(3)
		\Edge(3)(5)
		\tikzstyle{EdgeStyle}=[bend right]
		\Edge(0)(2)
	  	\Edge(2)(4)
 	\end{tikzpicture}
\end{figure}
\end{solution}

\section{Graphes hamiltoniens}
\subsection{Graphes hamiltoniens}
\index{chemin!chemin hamiltonien}
\begin{mydef}
  Un \emph{chemin} est \emph{hamiltonien} s’il passe par chaque noeud du graphe une et une seule fois.
\end{mydef}

\index{cycle!cycle hamiltonien}
\begin{mydef}
  Un \emph{cycle} est \emph{hamiltonien} s’il passe par chaque noeud du graphe une et une seule fois.
\end{mydef}

\index{graphe!graphe hamiltonien}
\begin{mydef}
  Un \emph{graphe} est \emph{hamiltonien} s’il possède un cycle hamiltonien.
\end{mydef}

\begin{mytheo} [Condition nécessaire pour un graphe hamiltonien]
  Si on ôte $k$ noeuds quelconques d’un graphe hamiltonien, on obtient au plus $k$ composantes connexes.
  \begin{proof}
     Soit $v_1 ... v_nv_1$, le cycle hamiltonien.\\
     Retirer $k$ noeuds du cycle laisse $\leq k$ composantes (=morceaux du cycle connexes), on crée au plus $k$ chemins, tous les noeuds sur un même chemin sont dans une même composante connexe. Les autres arêtes ne peuvent que diminuer encore le nombre de composantes connexes.
  \end{proof}
\end{mytheo}
\begin{myexem} Le graphe suivant est bien hamiltonien.
  \begin{figure} [!h]
      \hspace*{\fill}
         \subfigure[]
         {
             \begin{tikzpicture}[scale = 0.5]
          	 \fill[black] (-1,0) circle (0.1cm);
          	 \fill[black] (-2,2) circle (0.1cm);
          	 \fill[black] (-1.5,4) circle (0.1cm);
          	 \fill[black] (-2.5,5) circle (0.1cm);
          	 \fill[black] (0,6) circle (0.1cm);
          	 \fill[black] (2.5,4.5) circle (0.1cm);
          	 \fill[black] (2.5,2) circle (0.1cm);
          	 \fill[black] (3,1) circle (0.1cm);
          	 \fill[black] (1.5,0) circle (0.1cm);

          	 \draw (-1,0) -- (-2,2) -- (-1.5,4) -- (-2.5,5) -- (0,6) -- (2.5,4.5) -- (2.5,2) -- (3,1) -- (1.5,0) -- cycle;
        \end{tikzpicture}
      	}
      	\hfill
      \subfigure[] 
         {
             \begin{tikzpicture}[scale = 0.5]
          	 \fill[black] (-1,0) circle (0.1cm);
          	 \fill[black] (-2,2) circle (0.1cm);
          	 \fill[black] (-2.5,5) circle (0.1cm);
          	 \fill[black] (0,6) circle (0.1cm);
          	 \fill[black] (2.5,2) circle (0.1cm);
          	 \fill[black] (3,1) circle (0.1cm);

             \draw (-2.5,5) -- (0,6);
          	 \draw (-1,0) -- (-2,2);
          	 \draw (2.5,2) -- (3,1);
        \end{tikzpicture}
      	}
      	\hspace*{\fill}
    \end{figure} \newline
    En retirant 3 noeuds du graphe $g$ on obtient bien 3 composantes connexes dans le graphe $h$.
\end{myexem}

\begin{mytheo} [Condition suffisante pour un graphe hamiltonien]
  Un graphe simple à $n \geq 3$ noeuds tel que le degré minimum est d’au moins $n/2$ est hamiltonien.
  \begin{proof}
     Par l'absurde.

     Supposons qu'il existe un graphe à $n \geq 3$ noeuds de degré minimum $d(v) \geq \frac{n}{2}$ non hamiltonien.

     Prenons un tel graphe qui est maximal pour cette propriété : ajouter une arête dans ce graphe le rendrait hamiltonien.

     Ce graphe G n'est pas le graphe complet $K_n$
     \begin{align*}
		&\Rightarrow \exists \text{ des noeuds } v_1, v_n \text{ non adjacents} \\
		&\Rightarrow G + \{v_1, v_n \} \text{ est hamiltonien} \\
		&\Rightarrow \exists \text{ un cycle hamiltonien passant par l'arête } v_1v_n \\
		&\Rightarrow \text{Dans } G, \exists \text{ un chemin hamiltonien } v_1v_2...v_n
	\end{align*}
	Considérons les ensembles:
	\begin{align*}
		S &= \{v_i \mid v_{i+1} \text{ est adjacent à } v_1\} &\vert S \vert = \text{degré}(v_1) \geq \frac{n}{2}\\
		T &= \{v_i \mid v_{i} \text{ est adjacent à } v_n\} &\vert T \vert = \text{degré}(v_n) \geq \frac{n}{2}
	\end{align*}
	On sait que $v_n \not\in S$, par hypothèse $v_1$ et $v_n$ ne sont pas adjacents, et $v_n \not\in T$. On a donc $\vert S \cup T \vert < n$ puisqu'aucun des ensembles ne contient $v_n$.
	De plus, $\vert S \cap T \vert = \emptyset $ : en effet, si $\exists v_i \in \vert S \cap T \vert$, alors $v_1v_2...v_iv_nv_{n-1}...v_{i+1}v_1$ est un cycle hamiltonien $\Rightarrow$ Contradiction
	\begin{figure} [!h]
		\center
        \begin{tikzpicture}[scale = 0.5]
          	 \fill[black] (-7,0) circle (0.3cm);
          	 \node at (-7,-1) {$v_1$};
          	 \fill[black] (-5,0) circle (0.3cm);
          	 \node at (-5,-1) {$v_2$};
          	 \fill[black] (-1,0) circle (0.3cm);
          	 \node at (-1,-1) {$v_i$};
          	 \fill[black] (1,0) circle (0.3cm);
          	 \node at (1,1) {$v_{i+1}$};
          	 \fill[black] (5,0) circle (0.3cm);
          	 \node at (5,1) {$v_{n-1}$};
          	 \fill[black] (7,0) circle (0.3cm);
          	 \node at (7,1) {$v_n$};

          	 \draw (-7,0) -- (-5,0);
          	 \draw [dashed] (-5,0) -- (-1,0);
          	 \draw [dashed] (1,0) -- (5,0);
          	 \draw (5,0) -- (7,0);
          	 \draw (1,0) to[bend right] (-7,0);
          	 \draw (-1,0) to[bend right] (7,0);
        \end{tikzpicture}
    \end{figure}
	\newline \newline
	Enfin, $\vert S \cup T \vert = \vert S \vert + \vert T \vert - \vert S \cap T \vert \geq n$ mais $\vert S \cup T \vert < n$ $\Rightarrow$ Contradiction
  \end{proof}
\end{mytheo}

\index{problème}
\index{problème!du postier chinois}
\begin{mydef} [Problème du postier chinois]
  Dans un graphe pondéré, trouver le parcours fermé le plus court qui passe par toutes les arêtes au moins une fois.
\end{mydef}

\index{problème!du voyageur de commerce}
\begin{mydef} [Problème du voyageur de commerce]
  Dans un graphe pondéré, trouver le parcours fermé le plus court qui passe par tous les noeuds au moins une fois.
\end{mydef}

\section{Mariages, couplages et couvertures}
\subsection{Couplage}
\index{couplage}
\begin{mydef}
  Un \emph{couplage} dans un graphe est un ensemble $M$ d’arêtes tel que $M$ ne contient pas de boucles et deux arêtes de $M$ n’ont jamais d’extrêmité en commun.
\end{mydef}

\index{couplage!couplage maximum}
\begin{mydef}
  Un \emph{couplage maximum} est un couplage dont le nombre d’arêtes est maximal.
\end{mydef}

\index{couplage!couplage parfait}
\begin{mydef}
  Un \emph{couplage parfait} est un couplage qui est incident à tous les noeuds.
\end{mydef}

\begin{myrem}
  Un couplage parfait, s’il existe, est maximum.
\end{myrem}

\index{chemin!chemin M-alterné}
\begin{mydef}
  Pour un couplage $M$, un \emph{chemin M-alterné} est un chemin qui passe alternativement par une arête de $M$ et par une arête hors de $M$.
\end{mydef}

\index{chemin!chemin M-augmenté}
\begin{mydef}
  Un \emph{chemin M-augmenté} est un chemin M-alterné dont les noeuds d’origine et de destination ne sont pas incident à une arête de $M$.
\end{mydef}

\begin{mytheo} [Berge]
  Un couplage $M$ est maximum si et seulement s’il n’y a pas de chemin $M$-augmenté.
  \begin{proof}
     Preuve:
$ \Longrightarrow $ Soit le couplage $M$, représenté en bleu sur la figure ci-dessous, et un chemin $M$-augmenté en vert que nous noterons $P$. 
     
\begin{center}
    \begin{tikzpicture}[scale=1]
      \SetGraphUnit{1}
      \SetVertexNoLabel
      \Vertex{A}
      \EA(A){B}
      \EA(B){C}
      \SO(C){D}
      \WE(D){E}
      \EA(C){F}
      \EA(F){G}
      \SO(G){H}
      \Edges(C,D)
      \SetUpEdge[color=green]
      \Edges(A,B)
      \Edges(C,F)
      \Edges(G,H)
      \tikzset{EdgeStyle/.style={color=blue,thick,double=green,double distance = 1.2pt}}
      \Edges(F,G)
      \Edges(B,C)
      \SetUpEdge[color=blue]
      \Edges(D,E)
    \end{tikzpicture}
\end{center}        
     
     On construit le couplage $ M' = M \Delta P$ où $\Delta$ indique une différence symétrique entre $M$ et $P$ ($ M \Delta P = ( M \backslash P) \cup ( P \backslash M)$ ). Ce nouveau couplage est représenté en rouge.
     $ |M'| = |M| + 1 $. On voit bien que M n'est pas maximal.

\begin{center}
    \begin{tikzpicture}[scale=1]
      \SetGraphUnit{1}
      \SetVertexNoLabel
      \Vertex{A}
      \EA(A){B}
      \EA(B){C}
      \SO(C){D}
      \WE(D){E}
      \EA(C){F}
      \EA(F){G}
      \SO(G){H}
      \Edges(C,D)
      \Edges(F,G)
      \Edges(B,C)
      \SetUpEdge[color=red]
      \Edges(G,H)
      \Edges(C,F)
      \Edges(A,B)
      \Edges(D,E)
    \end{tikzpicture}
\end{center} 

     $\Longleftarrow$ Soit $M'$ le couplage maximum représenté ci-dessous tel que $ |M'| > |M| $. 

\begin{center}
\begin{tikzpicture}
\SetVertexNoLabel
\GraphInit[vstyle=Normal]
\SetGraphUnit{1}
\begin{scope}[rotate=90]
\Vertices{circle}{A,B,C,D}
\end{scope}
\begin{scope}[rotate=90,shift={(0,-3.5)}]
\Vertices{circle}{E,F,G,H}
\end{scope}
\begin{scope}[rotate=90,shift={(0,3.5)}]
\Vertices{circle}{I,J,K,L}
\end{scope}
\EA(H){M}
\Edges(F,G)
%\Edges(I,A)
\Edges(J,L,B,A,C,D,F,H,G,C)
\SetUpEdge[color=red]
\Edges(B,C)
\Edges(F,E)
\Edges(H,M)
\Edges(J,I)
\Edges(K,L)
\SetUpEdge[color=blue]
\Edges(J,K)
\Edges(I,L)
\Edges(E,H)
\tikzset{EdgeStyle/.style={color=blue,thick,double=red,double distance = 1.2pt}}
\Edges(A,D)
\end{tikzpicture}
\end{center}     
     
     Regardons $ M \Delta M'$. 
     
\begin{center}
\begin{tikzpicture}
\SetVertexNoLabel
\GraphInit[vstyle=Normal]
\SetGraphUnit{1}
\begin{scope}[rotate=90]
\Vertices{circle}{A,B,C,D}
\end{scope}
\begin{scope}[rotate=90,shift={(0,-3.5)}]
\Vertices{circle}{E,F,G,H}
\end{scope}
\begin{scope}[rotate=90,shift={(0,3.5)}]
\Vertices{circle}{I,J,K,L}
\end{scope}
\EA(H){M}
\SetUpEdge[color=red]
\Edges(B,C)
\Edges(F,E)
\Edges(H,M)
\Edges(J,I)
\Edges(K,L)
\SetUpEdge[color=blue]
\Edges(J,K)
\Edges(I,L)
\Edges(E,H)
\end{tikzpicture}
\end{center}
     
     On observe que les noeuds dans $ M \Delta M'$ dont de degrés $0,1$ ou $2$. Les noeuds de degrés $2$ ont une arête dans $M$ et une arête dans $M'$. $\Rightarrow$ $ M \Delta M'$ est une union de noeuds isolés, chemins et cycles. Or, dans un cycle de longueur paire, il y a autant d'arêtes dans $M$ que dans $M'$. Comme  $ |M'| > |M| $, il faut qu'il existe un chemin de longueur impaire qui commence et termine par une arête de $M'$. On voit facilement que ce chemin est un chemin $M$-augmenté.

  \end{proof}
\end{mytheo}
\begin{myexem}
  Exemple \addTODO
\end{myexem}

\begin{mytheo} [Théorème du mariage ou de Hall]
  Un graphe biparti avec bipartition $(X , Y)$ possède un couplage incident à tous les noeuds de $X$ si et seulement si pour tout ensemble $S \subseteq X$ , le nombre de voisins de $S$ est au moins $|S|$.
  \begin{proof}
     Preuve $\Longrightarrow$ Un couplage M incident à tout X crée une fonction injective: $X\mapsto Y$ donc $\forall S \subseteq X: Voisin (S) \geq M(S) $. On a donc que $ |Voisin (S)| \geq |M(S)| = |S| $.
     $\downarrow$ On veut montrer qu'il n'existe pas de couplage incident à tout X et donc qu'il existe un $ S \subseteq X: |voisin (S)| < |S|$. \\
     Soit $M*$ le couplage maximal et $ u \in X$ non incident à $M*$. Prenons les chemins $M*$ alternés partant de u: Soit z l'ensemble des noeuds ainsi rencontrés. $ S = X\cap Z$, $T= Y \cap Z$.
     \textit{Preuve.} On remarque \begin{itemize}
     \item Voisins (S)= T ( par constriuction)
     \item	$M*$ est incident à tout T (sinon on aurait un chemin M-augmenté car un chemin alterné qui part de u et qui arrive dans T peut toujours poursuivre par une arête de $M*$)
     \item $M*$ est incident à tout $S\backslash\lbrace u\rbrace$ par construction
     \item $M*$ crée une bijection entre $S\backslash\lbrace u\rbrace$ et T ( car $M* (S\backslash\lbrace u\rbrace) \subseteq T$ et $M*(T) \subseteq S\backslash\lbrace u\rbrace$.
     \end{itemize}
     $ \Rightarrow  Voisin (S)= T \Longleftrightarrow  |Voisin(s)| = |s|-1 < |s|$

  \end{proof}
\end{mytheo}
\begin{myexem}
  Exemple \addTODO
\end{myexem}

\begin{myrem}
  Un graphe est k-régulier si tous les noeuds sont de degré $k$.
\end{myrem}

\begin{mycorr}
  Tout graphe biparti k-régulier (pour $k > 0$) possède un couplage parfait.
\end{mycorr}

\subsection{Couverture}
\index{couverture de sommets}
\begin{mydef}
  Une \emph{couverture de sommets} d’un graphe est un ensemble de sommets incident à toutes les arêtes.
\end{mydef}

\index{couverture de sommets!minimum}
\begin{mydef}
  Une \emph{couverture de sommets minimum} d’un graphe est une couverture de sommets avec un nombre minimal de sommets.
\end{mydef}

\begin{myrem}
  Si $K$ est une couverture de sommets et $M$ un couplage, alors $|M| \leq |K|$.
\end{myrem}

\begin{myrem}
  Si $K^*$ est une couverture de sommets minimum et $M$ un couplage maximum, alors $|M^*| \leq |K^*|$.
\end{myrem}

\begin{mylem}
  Si $K$ est une couverture de sommets, $M$ un couplage et que $|M| = |K|$, alors $K$ est minimum et $M$ est maximum.
  \begin{proof}
     Preuve:
  \end{proof}
\end{mylem}

\begin{mytheo} [König]
  Dans un graphe biparti, si $K^*$ est une couverture de sommets minimum et $M^*$ un couplage maximum, alors $|M^*| = |K^*|$.
  \begin{proof}
     Preuve \addTODO
  \end{proof}
\end{mytheo}

\subsection{L'algorithme hongrois}
\index{algorithme!algorithme hongrois}
\begin{myalgo}[Algorithme hongrois]
  Algorithme \addTODO
\end{myalgo}
\begin{myexem}
  Exemple \addTODO
\end{myexem}

\section{Coloriages d'arêtes}
%il manque encore qq définitions

%dessin d'un graphe avec professeur et classes
\paragraph{Problème des horaires}
Chaque professeur doit enseigner à un certain nombre de classes pendant un certain nombre d'heures. On veut créer
un horaire sur le plus petit nombre de période possible
\\On relie chaque professeur aux classes auxquelles il donne cours en veillant a colorier les arêtes en fonction des tranches horaires. Deux arêtes de la même couleur ne peuvent pas partir du même nœud. 



\begin{mytheo}[König]
  Pour tout graphe biparti: $\chi '= degré max$
  \begin{proof}
    On va utiliser le théorème de Hall pour les graphes bipartis réguliers (qui ont toujours un couplage parfait).
    \begin{enumerate}
    
    
    \item Soit un graphe biparti $k$-régulier. Par le théorème de Hall, il existe un couplage parfait. On le colorie en couleur $c_{1}$. On considère ensuite les arêtes restantes non encore coloriées: elles forment un graphe $k-1$ régulier. On recommence pour la couleur $c_{2}$ avec un autre couplage. On continue jusqu'à épuisement, on obtient alors $\chi '=k$
    \item Pour un graphe biparti quelconque de degré $k$.
    \begin{itemize}
    \item Ajouter des nœuds d'un côté si nécessaire pour avoir le même nombre de nœuds de chaque côté.
    \item Si tous les nœuds ne sont pas de degré $k$, alors il y a au moins 1 nœuds de degré $<k$ de chaque côté. On ajoute alors une arête entre eux. On recommence jusqu'à $k$-régularité.
    \end{itemize}
    Par le point 1. , il existe un coloriage propre à $k$ couleurs. On supprime ensuite les arêtes et nœuds ajoutés: on obtient un coloriage propre pour le graphe de départ.
    $$\Rightarrow deg max \le \chi ' \le k=deg max$$
    $$\Rightarrow \chi ' = k$$
    \end{enumerate}
  \end{proof}
\end{mytheo}
\begin{mytheo} [Vizing]
Pour tout graphe: $\chi ' = deg max$ ou  $\chi ' = deg max + 1$
  \begin{proof} On sait que $\chi' \ge deg max$, il faut donc prouver que $\chi ' \le deg max + 1$.
  \\On le prouve par induction sur le nombre d'arêtes du graphe.
  \\ \textbf{Pas inductif:} Vrai pour $m$ arêtes. Soit un graphe à  $m+1$ arêtes, de degré max $k$. Je retire une de ces arêtes: il existe un coloriage propre à $\le k+1$ couleurs.
  \begin{itemize}
  \item Si $\le k$ couleurs: je choisis (k+1) couleurs pour la $(m+1)^{ième}$ arête.
  \item Si $k+1$ couleurs $c_{1},...,c_{k+1}$: je rétablis la $(m+1)^{ième}$ arête: il faut trouver une couleur pour cette arête.
  \end{itemize}

  \end{proof}
\end{mytheo}
%il faut encore ajouter les différents exemple
\begin{myexem}



\end{myexem}

% Ce fichier nécéssite les pacquets suivants
%\usepackage[utf8x]{inputenc}
%\usepackage{amsmath}
%\usepackage{amssymb}
%\usepackage{tikz}
%\usepackage{tkz-graph}
% et une définition de l'environnement «solution»
%
% Rédaction des réponses :
% Jacques Arnaud et Constant Matthieu

\section{Séance 7}

\subsection{confusion phonétique}
Lorsqu'on réalise un exposé oral, il faut choisir soigneusement ses notations afin d'éviter les confusions sonores entre lettres (comme entre $m$ et $n$ par exemple). Dans un alphabet de $N$ lettres, on a identifié toutes les paires de lettres qu'il est possible de confondre et on aimerait sélectionner un ensemble de lettres pour nos notations de sorte qu'aucune confusion ne soit possible.

\begin{enumerate}
  \item[a.] Modélisez cette situation comme un problème de graphes. Comment trouver le plus grand nombre de lettres non-confondables possible?
  \item[b.] On a besoin de $r+1$ symboles pour notre présentation. A partir de combien de couples de lettres confondables risque-t-on de devoir utiliser des symboles pouvant prêter à confusion?
\end{enumerate}

\begin{solution}
\begin{enumerate}
\item[a.]
Nous construisons un graphe $G$ dont les nœuds représentent les lettres et dont les arêtes représentent les confusions possibles. Pour trouver le plus grand nombre de lettres non-confondables, les nœuds isolés sont sélectionnés et dans chaque composante connexe, nous sélectionnons les nœuds formant un ensemble indépendant.

\item[b.]
Dans le pire cas, $G$ est connexe. Pour pouvoir sélectionner un ensemble de nœuds dans $G$, il faut que ces nœuds forment une clique dans le graphe complémentaire de $G$, que nous notons $\bar{G}$. Le théorème de Turán nous indique que si $\bar{G}$ a moins de $\left(1-\frac{1}{r}\right)\frac{N^2}{2}$ arêtes, alors $\bar{G}$ ne contient pas de clique. Ceci est vérifié lorsque $G$ a plus de $\frac{N(N-1)}{2}-\left(1-\frac{1}{r}\right)\frac{N^2}{2}$ arêtes.

\end{enumerate}
\end{solution}

\subsection{Groupe d'amis}
Dans un groupe de 9 personnes, une personne connaît 2 autres personnes, ces 2 personnes connaissent chacune 4 personnes, 4 personnes connaissent chacune 5 personnes et les 2 dernières personnes connaissent chacune 6 personnes. Montrez qu'il existe 3 personnes qui se connaissent l'une l'autre.

\begin{solution}
Nous commençons par calculer le nombre d'arêtes $m$ :

$$m=\frac{\sum(deg)}{2}=\frac{2+4+4+5+5+5+5+6+6}{2}=21$$

Le théorème de Turán nous permet de trouver une borne inférieure sur le nombre d'arêtes qui nous garantit une clique de taille 3:

$$\left(1-\frac{1}{2}\right)\frac{9^2}{2}=20.25$$
\end{solution}

\subsection{Big amitié}
Une fête regroupe $n$ personnes, chacune ayant au moins un ami présent (l'amitié étant réciproque). Dans tout groupe d'au moins 3 personnes, il n'y a jamais exactement 2 paires d'amis. Prouvez que chaque personne est amie avec toutes les autres.

\begin{solution}
Si l'on représente les amis par des nœuds et leurs amitiés par les arêtes, les seuls sous-graphes à 3 nœuds possibles (pour éviter d'avoir exactement deux paires d'amis dans un sous-graphe) sont les suivants:

\begin{center}
\begin{tikzpicture}
\GraphInit[vstyle=Normal]
\SetGraphUnit{1}
\begin{scope}[rotate=90]
\Vertices{circle}{$a$,$b$,$c$}
\end{scope}
\Edges($a$,$b$,$c$,$a$)
\end{tikzpicture}
\hspace*{2cm}
\begin{tikzpicture}
\GraphInit[vstyle=Normal]
\SetGraphUnit{1}
\begin{scope}[rotate=90]
\Vertices{circle}{$a$,$b$,$c$}
\end{scope}
\Edges($a$,$b$)
\end{tikzpicture}
\hspace*{2cm}
\begin{tikzpicture}
\GraphInit[vstyle=Normal]
\SetGraphUnit{1}
\begin{scope}[rotate=90]
\Vertices{circle}{$a$,$b$,$c$}
\end{scope}
\end{tikzpicture}
\end{center}

Si l'on prend un graphe à 3 nœuds, tout le monde est ami, car chaque personne a au moins un ami présent.

\begin{center}
\begin{tikzpicture}
\GraphInit[vstyle=Normal]
\SetGraphUnit{1}
\begin{scope}[rotate=90]
\Vertices{circle}{$a$,$b$,$c$}
\end{scope}
\Edges($a$,$b$,$c$,$a$)
\end{tikzpicture}
\end{center}

Ajoutons un nœud $d$, il doit forcément être connecté à un des nœuds existant déjà (par exemple le nœud $a$) puisque chaque personne a un ami présent.

\begin{center}
\begin{tikzpicture}
\GraphInit[vstyle=Normal]
\SetGraphUnit{1}
\begin{scope}[rotate=90]
\Vertices{circle}{$a$,$b$,$c$,$d$}
\end{scope}
\Edges($a$,$b$,$c$,$a$)
\SetUpEdge[color=green]
\Edges($a$,$d$)
\end{tikzpicture}
\end{center}

Dans ce nouveau graphe, tout ensemble de trois nœuds contenant $a$ et $d$ possède 2 arêtes puisque le troisième nœud est connecté à $a$. Ceci contredit l'hypothèse selon laquelle dans tout graphe d'au moins 3 personnes, il n'y a jamais exactement 2 paires d'amis, ce qui signifie qu'il faut relier $d$ au troisième nœud. Le même raisonnement s'applique à tout ensemble de 3 nœuds du graphe contenant $a$ et $d$. Par conséquent, $d$ doit être relié à tous les nœuds du graphe.

\begin{center}
\begin{tikzpicture}
\GraphInit[vstyle=Normal]
\SetGraphUnit{1}
\begin{scope}[rotate=90]
\Vertices{circle}{$a$,$b$,$c$,$d$}
\end{scope}
\Edges($a$,$b$,$c$,$a$)
\SetUpEdge[color=green]
\Edges($a$,$d$)
\SetUpEdge[color=red]
\Edges($b$,$d$,$c$)
\end{tikzpicture}
\end{center}

Cette preuve s'applique pour chaque nouvel ajout de nœuds au graphe. On en conclut que dans un graphe de $n$ nœuds, chaque personne est amie avec toutes les autres.
\end{solution}

\subsection{Graphe biparti}
Montrez que le graphe $G$ est biparti si et seulement si $\alpha(H) \geq \frac{1}{2} \nu(H)$ pour tout sous-graphe $H$ de $G$.

\begin{enumerate}
  \item[$\bullet$] $\alpha(H)$ est le nombre de sommets dans un ensemble indépendant maximum de $H$;
  \item[$\bullet$] $\nu(H)$ est le nombre de sommets de $H$.
\end{enumerate}

\begin{solution}
Considérons un sous graphe $H$ connexe de $G$ biparti. $H$ est lui même biparti (sinon $H$ perd sa connexité).
Chaque partition forme un ensemble indépendant de nœuds. Si les deux partitions de $H$ sont de tailles égales, le nombre de nœuds d'une partition est la moitié du nombre de nœuds de $H$, soit

$$\alpha(H)=\frac{1}{2}\nu(H)$$

Si une des deux partitions de $H$ est de taille plus grande, on obtient l'inégalité suivante

$$\alpha(H)>\frac{1}{2}\nu(H)$$

Considérons à présent un graphe $G$ dont les sous-graphes $H$ vérifient l'inégalité

$$\alpha(H)\geqslant\frac{1}{2}\nu(H)$$

Cette inégalité n'est vérifiée que si $G$ est biparti. En effet, supposons que $G$ comporte plus de deux ensembles indépendants distincts (autrement dit, plus de deux partitions). En sélectionnant un nœud dans chaque partition pour former un sous-graphe, il est facile de remarquer que l'inégalité n'est pas vérifiée pour ce sous-graphe (avec $a\geqslant3$ le nombre de partitions):

$$\alpha(H)=1\ngeqslant\frac{1}{2}\nu(H)=\frac{a}{2}$$
\end{solution}

\subsection{Graphe contenant un triangle}
Sans utiliser le Théorème de Turán, montrez que si le graphe $G = (V, E)$ est simple et que $|E| > \frac{|V|^2}{4}$, alors $G$ contient un triangle.

\begin{solution}
Le graphe, pour un nombre fixé de nœuds, possédant le plus d'arêtes tel qu'il n'existe pas de triangle formé par ses arêtes est un graphe biparti dont la taille des partitions diffère au maximum de 1 nœud.

Lorsque le nombre de nœuds $|V|$ est pair, les partitions sont de même taille $\frac{|V|}{2}$. Le nombre d'arêtes $|E|$ est $$|E|=\left(\frac{|V|}{2}\right)^2$$

Lorsque le nombre de nœuds $|V|$ est impair, les partitions sont de tailles $\frac{|V|+1}{2}$ et $\frac{|V|-1}{2}$. Le nombre d'arêtes $|E|$ est $$|E|=\frac{|V|+1}{2}~\frac{|V|-1}{2}=\frac{|V|^2-1}{4}$$

Le pire cas concernant le nombre d'arêtes, pour qu'un graphe ne contienne pas de triangle étant la première situation envisagée, la condition pour assurer qu'un graphe contienne au moins un triangle est $$|E|>\frac{|V|^2}{4}$$
\end{solution}

\section{Séance 8}

\textbf{Coloriage de sommets et polynôme chromatique}

\paragraph{1. } Quels sont les graphes de nombre chromatique égal à 1? et égal à 2?

\paragraph{2. } Trouvez le nombre chromatique du graphe de Pétersen, et du graphe biparti complet $K_{5,3}$.

\paragraph{3. } Vrai ou faux? \\
\begin{enumerate}[(a)]
  \item Un graphe de degré maximum 3 peut être colorié avec 4 couleurs.
  \item Un graphe de degré maximum 4 peut être colorié avec 4 couleurs.
  \item Si $G$ contient $K_n$ comme sous-graphe, alors son nombre chromatique est supérieur ou égal à $n$.
  \item Si $G$ est de nombre chromatique égal à $n$, alors $G$ contient $K_n$ comme sous-graphe.
\end{enumerate}



\paragraph{4. } L'algorithme glouton de coloration associé à un graphe $G$ fonctionne comme suit: on parcourt les sommets $v_1,v_2,…,v_n$ de $G$ dans un ordre fixé arbitrairement. Lorsqu'on rencontre le sommet $v_i$, on lui assigne la plus petite couleur qui n'est pas encore utilisée par un de ses voisins.
\begin{enumerate}[(a)]
  \item Montrez que tout graphe $G$ possède une séquence de sommets pour laquelle l'algorithme glouton utilise un nombre minimum de couleurs.
  \item Construisez pour tout $k \geq 1$ un arbre de degré maximum $k$ et une séquence de ses sommets pour laquelle l'algorithme glouton utilise $k+1$ couleurs.
\end{enumerate}




\paragraph{5. } Pour les deux graphes ci-dessous appliquez l'algorithme de coloration glouton, et trouvez $\chi(G)$.


\begin{figure}[h!]
  \centering
  %\begin{center}
  \subfigure[]{
    \begin{tikzpicture}[-,>=stealth',shorten >=1pt,auto]
      \Vertex[x=0 ,y=0]{1}
      \Vertex[x=1 ,y=1]{2}
      \Vertex[x=2,y=0]{3}
      \Vertex[x=1 ,y=-1]{4}
      \Vertex[x=3 ,y=1]{5}
      \Vertex[x=3 ,y=-1]{6}
      \Vertex[x=4 ,y=0]{7}



      \path[every node/.style={font=\sffamily\small}]
      (1) edge node [left] {} (2)
      edge node [left] {} (5)
      edge node [left] {} (3)
      edge node [left] {} (4)
      edge node [left] {} (6)

      (2) edge node [right] {} (3)
      edge node [right] {} (7)
      edge node [right] {} (5)

      (3) edge node [right] {} (4)
      edge node [left] {} (5)
      edge node [left] {} (6)
      edge node [left] {} (7)

      (4) edge node [right] {} (6)
      edge node [right] {} (7)

      (5) edge node [right] {} (7)

      (6) edge node [right] {} (7);

  \end{tikzpicture} }
  \subfigure[]{
    \begin{tikzpicture}[-,>=stealth',shorten >=1pt,auto]
      \Vertex[x=0 ,y=0]{1}
      \Vertex[x=1 ,y=1]{2}
      \Vertex[x=1,y=0]{3}
      \Vertex[x=1 ,y=-1]{4}
      \Vertex[x=3 ,y=1]{5}
      \Vertex[x=3 ,y=-1]{6}
      \Vertex[x=2 ,y=0]{7}

      \path[every node/.style={font=\sffamily\small}]
      (1) edge node [left] {} (2)
      edge node [left] {} (3)
      edge node [left] {} (4)

      (2) edge node [right] {} (3)
      edge node [right] {} (7)
      edge node [right] {} (5)

      (3) edge node [right] {} (4)
      edge node [left] {} (7)

      (4) edge node [right] {} (6)
      edge node [right] {} (7)

      (5) edge node [right] {} (7)
      edge node [right] {} (6)

      (6) edge node [right] {} (7);

    \end{tikzpicture}
  }

\end{figure}

\paragraph{6. } Sous quelle condition la somme des coefficients d'un polynôme chromatique est-elle nulle?

\paragraph{7. } Trouvez le polynôme chromatique du graphe biparti complet $K_{2,2}$, et donnez le nombre de colorations possibles du cycle $C_4$ avec 5 couleurs.

\paragraph{8. }
\begin{enumerate}[(a)]
  \item Trouvez le polynôme chromatique d'un graphe chemin de longueur $n$
  \item Utilisez ce résultat pour trouver le polynôme chromatique d'un graphe circuit de longueur $n$.
\end{enumerate}

\paragraph{9. } Soit le graphe parallélépipède $3 \times 3 \times 2$.

\begin{enumerate}[(a)]
  \item Chaque arête du graphe est de longueur 1. On souhaite décrire un chemin qui part du sommet $s$ et arrive au sommet $t$ en passant au moins une fois par chaque arête. Quelle est la longueur minimale d'un tel chemin?
  \item Quel est le nombre minimum de couleurs nécessaires pour colorier les sommets de façon à ce que deux sommets adjacents soient toujours de couleurs différentes?
  \item Soit $P_G(k)$ le polynôme chromatique de ce graphe. Quel est le degré de $P_G(k)$? Que vaut $P_G(2)$?
  \item Trouvez une expression aussi explicite que possible pour le nombre de chemins de longueur $k$ du sommet $s$ à lui-même.
\end{enumerate}

\begin{figure}[h!]
  \centering
  \includegraphics[scale=0.7]{img_tp8.jpg}
\end{figure}

\section{Graphes planaires}
\subsection{Formule d'Euler}
\begin{mytheo}
  \label{theo:threensix}
  Dans tout graphe planaire \emph{simple} à $n \geq 3$ sommets et $e$ arêtes,
  $e \leq 3n - 6$.
\end{mytheo}

\begin{mytheo}
  Pour tout graphe planaire \emph{simple}, il y a un noeud de degré $\leq 5$.
  \begin{proof}
    On va montrer que le degré moyen est $< 6$.
    Ce qu'il voudra dire qu'il existe un noeud de degré $\leq 5$.
    \begin{align*}
      \deg_{\mathrm{avg}} & = \frac{\sum_{v\in V} \deg(v)}{|V|}\\
                          & = \frac{2|E|}{|V|}.
    \end{align*}
    Considérons 2 cas
    \begin{itemize}
      \item Si $|V| < 3$, l'énoncé est trivial car dans un graphe simple,
        pour tout $v \in V$, $\deg(v) \leq |V|-1$ du coup
        $\deg(v) \leq |V| - 1 < 2 \leq 5$ pour tout $v$.
      \item
        Comme notre graphe est simple,
        on peut utiliser le théorème~\ref{theo:threensix},
        on a donc $|E| \leq 3|V| - 6$.
        Dès lors
        \begin{align*}
          \deg_{\mathrm{avg}} & \leq 2\frac{3|V|-6}{|V|}\\
                              & = 6 - \frac{12}{|V|} < 6.
        \end{align*}
    \end{itemize}
  \end{proof}
\end{mytheo}

\begin{mytheo}
  $K_5$ est non planaire.
  \begin{proof}
    $K_5$ a 5 noeuds et 10 arêtes.
    Par le théorème~\ref{theo:threensix}, $|E| \leq 3|V| - 6$.
    Il faut donc que $10 \leq 3 \cdot 5 - 6 = 9$, ce qui est faux.
    Le graphe est par conséquent non planaire.
  \end{proof}
\end{mytheo}

\begin{mytheo}
  $K_{3,3}$ est non planaire.
  \begin{proof}
    $K_{3,3}$ a 6 noeuds et 9 arêtes.
    C'est un graphe biparti donc les cycles sont de longueur pair de plus il est simple donc tous les cycles ont une longueur $\geq 4$.
    Donc toutes les faces ont un degré $\geq 4$.
    On a alors $\sum_{f \in F} \deg(f) \geq 4|F|$ et par le théorème des poignées de main dual, $\sum_{f \in F} \deg(f) = 2|E| = 18$.
    Donc $|F| \leq \frac{18}{4} = 4.5$.

    Par la formule d'Euler, il faut que
    $|F| - |E| + |V| = 2$.
    Or $|F| - |E| + |V| \leq 4.5 - 9 + 6 = 1.5$.
    $K_{3,3}$ ne peut donc pas être planaire.
  \end{proof}
\end{mytheo}

\begin{mydef}
  Un solide platonicien est un polyèdre convexe régulier.
  C'est-à-dire que toutes les faces, sommets et arêtes sont identiques à une rotation près.
\end{mydef}

\begin{mytheo}
  Il y a 5 solides platoniciens.
  \begin{proof}
    Les polyèdres convexes correspondent à des graphes planaires, via projection.
    Le fait qu'ils soient platoniciens nous dit que chaque noeud est de même degré $p$ et que chaque face est de même degré $q$.
    \begin{center}
      \begin{tabular}{ll}
        La formule d'Euler & $|F| - |E| + |V| = 2$\\
        Poignées de main & $p|V| = 2|E|$\\
        Poignées de main dual & $q|F| = 2|E|$
      \end{tabular}
    \end{center}
    Donc
    \begin{align*}
      \frac{2}{q}|E| - |E| + \frac{2}{p} |E| & = 2\\
      \frac{2}{q} - 1 + \frac{2}{p} & = \frac{2}{|E|} > 0\\
      \frac{1}{q} + \frac{1}{p} & > \frac{1}{2}.
    \end{align*}
    On sait donc que soit $p$, soit $q$ est $< 4$ (ou les deux).
    Or $p \geq 3$ et $q \geq 3$ (par géométrie, graphes planaires simple de dual simple).
    Les possibilités sont
    \begin{center}
      \begin{tabular}{|c|c|c|c|c|c|}
        \hline
        $p$ & $q$ & $|V|$ & $|F|$ & $|E|$ & Polyèdre\\
        \hline
         3  &  3  &   4   &   4   &   6   & Tétraèdre\\
         3  &  4  &   8   &   6   &  12   & Cube\\
         4  &  3  &   6   &   8   &  12   & Octaèdre\\
         3  &  5  &  20   &  12   &  30   & Dodécaèdre\\
         5  &  3  &  12   &  20   &  30   & Icosaèdre\\
        \hline
      \end{tabular}
    \end{center}
  \end{proof}
\end{mytheo}

\begin{mytheo}
  ``Toute carte peut être coloriée avec 5 couleurs''.
  Nombre chromatique $\chi$ (graphe planaire) $\leq 5$.
  \begin{proof}
    Par récurrence: ``on enlève un noeud, on colorie par hyp. de récurrence, on remet le noeud.''
    On peut supposer le graphe \emph{simple} (car arêtesmultiples n'affectent pas $\chi$).
    Il existe un noeud $u$ de degré $\leq 5$.
    On enlève $u$, on a encore un graphe planaire, on le colorie.
    On rétablit $u$:
    \begin{itemize}
      \item si $\deg(u) < 5$: facile, on utilise une couleur non utilisée par les voisins pour $u$.
      \item Si $\deg(u) = 5$: Si ces 5 voisins utilisent $< 5$ couleurs: facile aussi.
        Si 5 couleurs utilisées $c_1, c_2, c_3, c_4, c_5$.
        Regardons $v_1$ et $v_3$. Si $v_1$ et $v_3$ sont sur des composantes connexes différentes: on échange $c_1$ et $c_3$ sur
        la composante connexe ($c_1-c_3$) de $v_3$, et on colorie $u$ en $c_1$ (sur le graphe des noeuds de couleur $c_1$ et $c_3$.
        Si $v_1$ et $v_3$ sont dans la même composante connexe ($c_1-c_3$):
        Maintenant $v_2$ et $v_4$ sont dans des composantes connexes
        différentes (dans le graphe de couleurs $c_2-c_4$).
        Même raisonnement: échanger $c_2$ et $c_4$ sur composante connexe ($v_2$).
    \end{itemize}
  \end{proof}
\end{mytheo}

\section{Séance 10}

\paragraph{1. } Déterminez le flot maximum pour le réseau ci-dessous. Comment montrer que la solution proposée est bien optimale? 

\begin{figure}[h!]
  \centering
  \begin{tikzpicture}
    \SetGraphUnit{3}
    \GraphInit[vstyle=Dijkstra]
    \SetUpEdge[style={->},
    labelstyle = {sloped,draw}]
    \SetVertexNoLabel
    \Vertex[NoLabel=false]{S}
    \NOEA(S){A} \SOEA(A){O} \SOEA(S){B}
    \NOEA(O){C} \SOEA(O){D} \EA(O){E}
    \EA[NoLabel=false](E){T}
    \Edge[label=$6$](S)(A)
    \Edge[label=$4$](S)(B)
    \Edge[label=$9$](S)(O)
    \Edge[label=$3$](A)(C)
    \Edge[label=$6$](B)(D)
    \Edge[label=$1$](C)(O)
    \Edge[label=$8$](C)(T)
    \Edge[label=$1$](D)(E)
    \Edge[label=$6$](D)(T)
    \Edge[label=$1$](E)(B)
    \Edge[label=$2$](E)(C)
    \Edge[label=$4$](E)(T)
    \Edge[label=$1$](O)(D)
    \Edge[label=$8$](O)(E)
    \tikzset{EdgeStyle/.append style = {bend left}}
    \Edge[label=$2$](A)(B)
    \Edge[label=$1$](B)(A)
    \tikzset{EdgeStyle/.append style = {bend right}}
    \Edge[label=$2$](B)(C)
  \end{tikzpicture}
\end{figure}

\begin{solution}
  On voit que $f_\mathrm{net}(S) = -f_\mathrm{net}(T) = 17$.
  Si on essaie de trouver un chemin augmentant avec un BFS ou DFS,
  en s'autorisant donc à prendre,
  \begin{itemize}
    \item Soit les arêtes non-saturées.
    \item Soit les arêtes telles qu'il y ait une arête dans l'autre
      sens avec un flot non-null (back edges).
  \end{itemize}
  On part de $S$ mais on arrive jamais à $T$.
  On a donc $f_\mathrm{max} = 17$.
  \begin{center}
    \begin{tikzpicture}
      \SetGraphUnit{3}
      \GraphInit[vstyle=Dijkstra]
      \SetUpEdge[style={->},
      labelstyle = {sloped,draw}]
      \SetVertexNoLabel
      \Vertex[NoLabel=false]{S}
      \NOEA(S){A} \SOEA(A){O} \SOEA(S){B}
      \NOEA(O){C} \SOEA(O){D} \EA(O){E}
      \EA[NoLabel=false](E){T}
      \Edge[label=$5/6$](S)(A)
      \Edge[label=$4/4$](S)(B)
      \Edge[label=$8/9$](S)(O)
      \Edge[label=$3/3$](A)(C)
      \Edge[label=$5/6$](B)(D)
      \Edge[label=$0/1$](C)(O)
      \Edge[label=$7/8$](C)(T)
      \Edge[label=$0/1$](D)(E)
      \Edge[label=$6/6$](D)(T)
      \Edge[label=$1/1$](E)(B)
      \Edge[label=$2/2$](E)(C)
      \Edge[label=$4/4$](E)(T)
      \Edge[label=$1/1$](O)(D)
      \Edge[label=$7/8$](O)(E)
      \tikzset{EdgeStyle/.append style = {bend left}}
      \Edge[label=$2/2$](A)(B)
      \Edge[label=$0/1$](B)(A)
      \tikzset{EdgeStyle/.append style = {bend right}}
      \Edge[label=$2/2$](B)(C)
    \end{tikzpicture}
  \end{center}
\end{solution}


\paragraph{2. } Soit le réseau représenté plus bas. Votre patron est convaincu qu'il est possible de faire passer 138 unités de flots de $s$ à $t$ et il vous reproche de ne pas être capable d'exhiber un tel flot. Convainquez votre patron par un argument simple qu'un tel flot n'existe pas. Quelle est la valeur maximale d'un flot dans ce réseau? 

\begin{figure}[h!]
\centering
\includegraphics[scale=0.5]{graphTP10_1.pdf}
\end{figure}

\paragraph{3. } Dans un petit village en Ardenne, il y a $n$ célibataires masculins, $n$ célibataires féminins et $m$ entremetteuses. Chaque entremetteuse connait certains des célibataires masculins ainsi que certains des célibataires féminins. De plus, l'entremetteuse $i$ ne peut arranger qu'au plus $b_i$ mariages entre les célibataires qu'elle connait. On suppose que seuls les mariages hétérosexuels sont acceptés et que les célibataires ne se marient qu'une fois. On souhaite déterminer le nombre maximum de mariages qui peuvent être arrangés. Montrez comment ce problème peut être formulé comme un problème de flot maximum dans un graphe. 

\paragraph{4. } Vous disposez d'une machine parallèle de deux processeurs A et B de types différents sur laquelle vous devez faire tourner une série de $n$ processus. Chaque processus doit être attribué à un processeur. Si vous choisissez d'assigner le processus $i$ au processeur A, cela induit un coût $\alpha_i$, alors que si vous choisissez le B, le coût est $\beta_i$. De plus, il est avantageux de faire tourner certains processus sur le même processeur parce que les transferts de données entre les deux sont élevés. Le coût d'assigner les processus $i$ et $j$ à des processeurs différents est égal à $c_{ij}$. Déterminez l'attribution optimale pour les données ci-dessous. 

\begin{multicols}{2}

\begin{center}
\begin{tabular}{||c||c|c|c|c||}
\hline
$i$ & 1 & 2 & 3 & 4 \\ 
\hline
$\alpha_i$ & 6 & 5 & 10 & 4 \\
\hline
$\beta_i$ & 4 & 10 & 3 & 8 \\
\hline
\end{tabular}
\end{center}

\columnbreak


\begin{center}
\begin{tabular}{||c||c|c|c|c||}
\hline
$c_{ij}$ & 1 & 2 & 3 & 4 \\ 
\hline
1 & 0 & 5 & 0 & 0 \\
\hline
2 & 5 & 0 & 6 & 2 \\
\hline
3 & 0 & 6 & 0 & 1 \\
\hline 
4 & 0 & 2 & 1 & 0 \\
\hline
\end{tabular}
\end{center}

\end{multicols}

\paragraph{5. } L'algorithme de Ford-Fulkerson consiste à trouver à chaque itération un chemin d'augmentation et d'augmenter le flot le long de ce chemin jusqu'à saturation. Montrez par un argument élémentaire que dans le cas où les capacités maximales sont entières cet algorithme s'arrête toujours après un nombre fini d'itérations. Pouvez-vous adapter votre argument à la situation pour laquelle les capacités sont rationnelles? \\
\textbf{Challenge:} et pour des capacités réelles? 

\paragraph{6. } Un laboratoire a la possibilité de réaliser 6 projets. La réalisation du projet $i$ génère un revenu $q_i$. Pour réaliser le projet $i$, le laboratoire a besoin d'une présence de l'ensemble des chercheurs $T_i$. Engager le chercheur $j$ coûte $p_j$. Etant donné les revenus $q= (8,7,4,5,3,9)$, les coûts $p=(2,10,7,3,5)$ et $T_1 = \left\lbrace 1, 4 \right\rbrace$, $T_2 = \left\lbrace 1,2,3,4 \right\rbrace$, $T_3 = \left\lbrace 1,3,5 \right\rbrace$, $T_4 = \left\lbrace 4,5 \right\rbrace$, $T_5 = \left\lbrace 3,4,5 \right\rbrace$, $T_6 = \left\lbrace 1, 4,5 \right\rbrace$, trouvez l'ensemble des projets qui maximise le profit total (revenus moins coûts).

\newpage
\textbf{Exercices supplémentaires}

\paragraph{7. } Un producteur désire envoyer au départ des sommets $D_1$ et $D_2$ un produit aux destinations $M_1, M_2$, et $M_3$ à travers le réseau. Les capacités des arêtes sont limitées. Il y a des demandes respectives de 10, 8 et 8 unités aux destinations $M_1, M_2,$ et $M_3$. Ces demandes peuvent-elles être satisfaites? 

\begin{figure}[h!]
\centering
\includegraphics[scale=0.5]{graphTP10_2.pdf}
\end{figure}

\paragraph{8. } On considère un groupe de six personnes $\left\lbrace A, B, … , F \right\rbrace$ qui sont toutes membres d'un certain nombre de clubs de sport $\left\lbrace 1, 2, 3, 4, 5, 6 \right\rbrace$. Ci-dessous nous donnons la liste des membres de chaque club. \\
Club 1: A, C \\
Club 2: C, E \\
Club 3: A, B, C, D, E, F \\
Club 4: A, C, E \\
Club 5: A, B, D, E, F \\
Club 6: A, E\\
On souhaite choisir un représentant pour chaque club. Pour être représentant d'un club, il faut en être membre. Par ailleurs, une même personne ne peut représenter qu'un seul club. Est-il possible de trouver des représentants pour tous les clubs? Justifiez votre réponse. Proposez une solution dans le cas d'un ensemble $M$ de personnes, avec $S_j \subseteq M$ la liste des membres du club $j$. 


\end{document}
