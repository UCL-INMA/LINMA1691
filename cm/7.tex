\section{Cliques, ensembles indépendants et l'impossible désordre}
\subsection{Ensembles indépendants}
\begin{mytheo} [Théorème de l'amitié]
  Parmi six personnes, on en trouve toujours trois qui se connaissent l’une l’autre, ou trois qui sont étrangers l’un à l’autre.
  \begin{proof}
     Preuve \textcolor{red}{TODO}
  \end{proof}
\end{mytheo}

\index{ensemble indépendant}
\begin{mydef}
  Un \emph{ensemble indépendant} d’un graphe est un ensemble de noeuds deux à deux non adjacents.
\end{mydef}

\index{ensemble indépendant!maximum}
\begin{mydef}
  Un \emph{ensemble indépendant maximum} est un ensemble indépendant dont le nombre de noeuds est maximal.
\end{mydef}

\begin{mytheo}
  Un ensemble de noeuds est indépendant si et seulement si son complémentaire est une couverture de sommets.
  \begin{proof}
     Preuve \textcolor{red}{TODO}
  \end{proof}
\end{mytheo}

\begin{mycorr}
  $|$ensemble indépendant maximum$| + |$couverture minimum$| =
|$nombre de noeuds$|$
  \begin{proof}
     Preuve \textcolor{red}{TODO}
  \end{proof}
\end{mycorr}

\subsection{Cliques}
\index{clique}
\begin{mydef}
  Une \emph{clique} d’un graphe est un ensemble de noeuds deux à deux adjacents. Autrement dit, c’est un sous-graphe complet.
\end{mydef}

\index{clique!maximum}
\begin{mydef}
  Une \emph{clique maximum} est une clique dont le nombre de noeuds est maximale.
\end{mydef}

\begin{mytheo}
  Un ensemble est indépendant dans un graphe simple si et seulement s’il est une clique dans le graphe complémentaire.
  \begin{proof}
     Preuve \textcolor{red}{TODO}
  \end{proof}
\end{mytheo}

\begin{mytheo} [Théorème de l'amitié]
  Tout graphe simple à six noeuds contient une clique de trois noeuds ou un ensemble indépendant de trois noeuds.
  \begin{proof}
     Preuve \textcolor{red}{TODO}
  \end{proof}
\end{mytheo}

\begin{mytheo} [Théorème de l'amitié]
  En coloriant, de façon arbitraire, les arêtes du graphe complet à six noeuds en bleu et rouge, on crée un triangle bleu ou un triangle rouge.
  \begin{proof}
     Preuve \textcolor{red}{TODO}
  \end{proof}
\end{mytheo}

\begin{mytheo} [Théorème de Ramsey]
  Soit un graphe complet à $r$ noeuds. On colorie les arêtes en les couleurs $c_1$ , ..., $c_k$ . On cherche la plus petite valeur de $r$ tel que tout coloriage crée une clique à $n_1$ noeuds de couleur $c_1$ , ou une clique à $n_2$ noeuds de couleur $c_2$ , ..., ou une clique à $n_k$ noeuds de couleur $c_k$ . Cette plus petite valeur de $r$, est le nombre de Ramsey $R(n_1 , ..., n_k)$.\\
  $R(n_1 , ..., n_k)$ existe !
  \begin{proof}
     Preuve \textcolor{red}{TODO}
  \end{proof}
\end{mytheo}

\begin{mytheo} [Théorème de Erdös et Szekeres]
  Pour $m, n \geq 2: R(m, n) \leq R(m, n-1) + R(m-1, n)$.
  \begin{proof}
     Preuve \textcolor{red}{TODO}
  \end{proof}
\end{mytheo}

\begin{mycorr}
  $R(m, n) \leq (
    \begin{array}{c}
      m+n-2 \\
      m-1
    \end{array})$.
  \begin{proof}
     Preuve \textcolor{red}{TODO}
  \end{proof}
\end{mycorr}

\begin{mytheo}
  $R(n_1, ..., n_k) \leq R(n_1, ..., n_{k-2}, R(n_{k-1}, n_k))$.
  \begin{proof}
     Preuve \textcolor{red}{TODO}
  \end{proof}
\end{mytheo}

\begin{mytheo} [Théorème de l'amitié]
  $R(3, 3) = 6$
  \begin{proof}
     Preuve \textcolor{red}{TODO}
  \end{proof}
\end{mytheo}

\begin{mytheo} [Théorème de Turán]
  Si un graphe simple a strictement plus de $(1 − \frac{1}{r}) \frac{n^2}{2}$ arêtes, alors il a une clique de $r + 1$ noeuds.
  \begin{proof}
     Preuve \textcolor{red}{TODO}
  \end{proof}
\end{mytheo}

\begin{mytheo} [Théorème de Schur]
  Pour chaque $k$ , il y a un nombre $r_k$ tel que pour toute partition des nombres $1, 2, ..., r_k$ en $k$ classes, une de ces classes contient $x, y , z$ tels que $x + y = z$.
  \begin{proof}
     Preuve \textcolor{red}{TODO}
  \end{proof}
\end{mytheo}

\begin{mytheo} [Théorème de Esther Klein]
  Parmi cinq points arbitraires dans le plan, tels que trois d’entre eux ne sont jamais alignés, on peut toujours en choisir quatre qui déterminent un quadrilatère convexe.
  \begin{proof}
     Preuve \textcolor{red}{TODO}
  \end{proof}
\end{mytheo}

\begin{mytheo} [Théorème de Van der Waerden]
  Pour tout $k , l,$ il existe un nombre $W (k , l)$ tel que les nombres de $1$ à $W (k , l)$, coloriés arbitrairement en $k$ couleurs, contiennent une progression arithmétique monochrome de longueur $l$.
  \begin{proof}
     Preuve \textcolor{red}{TODO}
  \end{proof}
\end{mytheo}







