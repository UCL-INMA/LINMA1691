\section{Arbres et connectivité}
\subsection{Arbres}
\begin{mydef}
  Un \emph{arbre} est un graphe connexe et sans cycle. Une \emph{forêt} est un graphe sans cycle.
\end{mydef}

\begin{mydef}
  Un sous-graphe sous-tendant ou couvrant d’un graphe $G$ est un sous-graphe qui contient tous les sommets de $G$.
\end{mydef}

\begin{mytheo} [Arbres sous-tendants]  
  Tout graphe connexe contient un arbre sous-tendant.
  \begin{proof}
     \href{https://dl.dropboxusercontent.com/u/44092863/Graph_Theory_Romain_Capron.pdf}{Voir notes}
  \end{proof}
\end{mytheo}

\begin{mytheo} [Caractérisations des arbres]
  Soit $G$ un graphe à $n$ sommets et $m$ arêtes. Alors les conditions suivantes sont équivalentes :
  \begin{itemize}
    \item $G$ est connexe et sans cycle;
    \item $G$ est sans cycle et $m = n − 1$;
    \item $G$ est connexe et $m = n − 1$;
    \item $G$ est connexe et supprimer une arête quelconque déconnecte $G$;
    \item $G$ est sans cycle et ajouter une arête quelconque crée un et un seul cycle;
    \item Deux noeuds de $G$ sont toujours reliés par un seul chemin.
  \end{itemize}
  \begin{proof}
     \href{https://dl.dropboxusercontent.com/u/44092863/Graph_Theory_Romain_Capron.pdf}{Voir notes}
  \end{proof}
\end{mytheo}