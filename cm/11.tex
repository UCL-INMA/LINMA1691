\section{$\mathcal{P}$, $\mathcal{NP}$ et $\mathcal{NP}$-complétude}
\subsection{$\mathcal{P}$, $\mathcal{NP}$ et $\mathcal{NP}$-complétude}
\index{problème!problème de décision}
\index{instance}
\begin{mydef}
  Un \emph{problème de décision} est une liste d’objets finis appelés \emph{instances}, chacune porteuse d’un label OUI ou NON. Donc c’est une fonction des instances vers l’ensemble \{OUI, NON\}. Les instances positives sont celles de label OUI.\\
  Un algorithme résout un problème de décision donné s’il prend en entrée une instance du problème et renvoie en sortie le label OUI ou NON correspondant à l’instance.
\end{mydef}

\index{instance!taille d'une instance}
\begin{mydef}
  La \emph{taille d’une instance} est le nombre de bits nécessaire pour décrire l’instance. Pour un graphe, on considère souvent que c’est son nombre d’arêtes ou son nombre de noeuds.
\end{mydef}

\index{algorithme!algorithme efficace}
\index{algorithme!algorithme polynomial}
\begin{mydef}
  Un algorithme est \emph{efficace} ou \emph{polynomial} s’il produit une réponse en un temps borné par un polynôme de la taille de l’entrée.
\end{mydef}

\index{problème!classe $\mathcal{P}$}
\begin{mydef}
  La \emph{classe $\mathcal{P}$} est la classe des problèmes pour lesquels il existe un algorithme efficace qui les résout.\\
  Cette classe contient par exemple le problème du graphe eulérien.
\end{mydef}

\index{problème!classe $\mathcal{NP}$}
\begin{mydef}
  Un problème est dans $\mathcal{NP}$ s’il existe un algorithme efficace $A$ tel que :
  \begin{itemize}
    \item pour chaque instance positive $m$, il existe un objet fini $n$ tel que $A(m, n)$ renvoie OUI,
    \item pour chaque instance négative $m$ et tout objet fini $n$, $A(m, n)$ renvoie NON.
  \end{itemize}
\end{mydef}

\begin{myexem}
  Problème du graphe hamiltonien:
  \begin{align*}
    m & = \text{graphe non-dirigé} = \text{instance}\\
    n & = \text{cycle dans }m = \text{``preuve'' ou ``témoin''}\\
    A(m,n) & = \text{OUI ssi }n\text{ est hamiltonien}.
  \end{align*}
\end{myexem}

\begin{mycorr}
  $\mathcal{P} \subseteq \mathcal{NP}$
  \begin{proof}
     C'est par définition de l'ensemble. On verra à la fin de cette section qu'il pourrait (bien qu'il est peu probable) prouver que $\mathcal{P}=\mathcal{NP}$
  \end{proof}
\end{mycorr}

\index{problème!problème réductible}
\begin{mydef}
  Etant donnés deux problèmes $P$ et $Q$, on dit que $P$ est \emph{réductible en temps polynomial} à $Q$, ou que $P$ est \emph{réductible} à $Q$, ou plus informellement que $P$ est plus facile que $Q$, noté $P \leq Q$, s’il existe un algorithme efficace $A$ (appelé réduction) qui prenne en entrée une instance $n$ de $P$ et renvoie en sortie une instance $A(n)$ de $Q$, telle que $n$ est positive ssi $A(n)$ est positive.
\end{mydef}

\begin{myexem}
  Soit les problèmes suivant
  \begin{description}
    \item{Problème} COUPLAGE-PARFAIT
    \item[Instance] Graphe non-dirigé
    \item[Question] Existe-t-il un couplage parfait ?
  \end{description}
  \begin{description}
    \item{Problème} POLYÈDRE-NON-VIDE
    \item[Instance] Inégalité $Ax \leq b$ (géométriquement : polyèdre $\{x \in \mathbb{R}^m : Ax \leq b\}$)
    \item[Question] Le polyèdre est-il non vide ? C'est à dire, $\exists x : Ax \leq b$ ?
  \end{description}
  COUPLAGE-PARFAIT $\leq$ POLYÈDRE-NON-VIDE
  \begin{proof}
    Soit $M$ la matrice d'incidence du graphe.
    Il existe un couplage parfait si et seulement si $\exists x \in \{0,1\}^m$ tel que
    $Mx = \begin{pmatrix}
      1\\
      \vdots\\
      1
    \end{pmatrix}$ si et seulement si (il faut admettre le sens ``seulement si'', le sens ``si'' est trivial)
    $\exists x \in \mathbb{R}^n$ tel que $Mx = \begin{pmatrix}
      1\\
      \vdots\\
      1
    \end{pmatrix}$ et $0 \leq x_i \leq 1$.
    Ce qui est un cas particulier de la question
    ``$\exists x : Ax \leq b$ ?''
    Fabriquer $A$ et $b$ à partir du graphe se fait en temps polynomial en la taille du graphe.
  \end{proof}
  Donc: Tout algo pour résoudre POLYÈDRE-NON-VIDE peut être utilisé pour résoudre COUPLAGE-PARFAIT.
\end{myexem}

\begin{mytheo}
  Si $P \leq Q$ et $Q \in \mathcal{P}$ alors $P \in \mathcal{P}$.
  \begin{proof}
     Cette preuve n'a pas été vue.
  \end{proof}
\end{mytheo}

\begin{myexem}
  Soit les problèmes suivant
  \begin{description}
    \item{Problème} $k$-CLIQUE
    \item[Instance] Graphe
    \item[Question] Existe-t-il une clique à $k$ noeuds ?
  \end{description}
  $k$-CLIQUE $\leq$ $k$-INDEPENDANT $\leq$ $k$-CLIQUE.
  Ils sont donc équivalents.
\end{myexem}

\index{problème!problème $\mathcal{NP}$-complet}
\begin{mydef}
  On dit qu'un problème $P$ est $\mathcal{NP}$-complet si $P \in \mathcal{NP}$ et pour tout problème $Q \in \mathcal{NP}$, $Q \leq P$.\\
  C'est donc un problème de $\mathcal{NP}$ plus difficile que tous les autres.
\end{mydef}

\begin{mytheo} [Cook]
  SAT est $\mathcal{NP}$-complet.
\end{mytheo}

\begin{myexem}
  Soit le problème suivant
  \begin{description}
    \item{Problème} SAT
    \item[Instance] Formule logique
    \item[Question] Existe-t-il des valeurs de vérité pour les variables telles que la formule est vraie.
  \end{description}
  Par exemple, l'instance $\lnot(P \lor Q) \land (R \lor \lnot Q)$ est une instance positive car
  $(R, P, Q) = (\text{VRAI}, \text{FAUX}, \text{FAUX})$ vérifie la formule.
\end{myexem}

\begin{mytheo} [Karp]
  Déterminer si un graphe est hamiltonien est $\mathcal{NP}$-complet.
  \begin{proof}\hfill
    \begin{itemize}
      \item HAMILTONIEN $\in \mathcal{NP}$: trivial.
      \item Il existe une réduction de SAT vers HAMILTONIEN (partie difficile de la preuve) donc
        SAT $\leq$ HAMILTONIEN.
    \end{itemize}
    Donc HAMILTONIEN est $\mathcal{NP}$-complet.
    On a d'ailleurs SAT $=$ HAMILTONIEN car SAT est $\mathcal{NP}$-complet.
  \end{proof}
\end{mytheo}

\begin{mytheo}
Comment prouver qu'un problème $P$ est $\mathcal{NP}$-complet?
\begin{proof}\hfill
  \begin{enumerate}
    \item Prouver $P \in \mathcal{NP}$;
    \item Choisir un problème $Q$ connu comme $\mathcal{NP}-complet$;
    \item Prouver $Q \leq P$ en exhibant une réduction efficace.
  \end{enumerate}
  \end{proof}
\end{mytheo}

\begin{mydef}
On pourrait essayer de prouver que $\mathcal{P}=\mathcal{NP}$ si et seulement si on montrait qu'un problème $\mathcal{NP}$  (Hamilton, SAT ou encore le sudoku) $\in$ P
\end{mydef}

\begin{myexem}
  Quelques problèmes $\mathcal{NP}$-complets en théorie des graphes :
  \begin{itemize}
    \item $k$-CLIQUE
    \item $k$-COUVERTURE
    \item $k$-INDEPENDANT
    \item HAMILTONIEN
    \item $k$-COLORABILITE
    \item $k$-ARETE-COLORABILITE
    \item COUPE-MAX
    \item VOYAGEUR DE COMMERCE
  \end{itemize}
\end{myexem}

