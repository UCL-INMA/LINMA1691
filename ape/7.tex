\section{Séance 7}

\paragraph{1. } Lorsqu'on réalise un exposé oral il faut choisir soigneusement ses notations afin d'éviter les confusions sonores entre lettres (comme entre $m$ et $n$ par exemple). Dans un alphabet de $N$ lettres, on a identifié toutes les paires de lettres qu'il est possible de confondre et on aimerait sélectionner un ensemble de lettres pour nos notations de sorte qu'aucune confusion ne soit possible.
\begin{enumerate}
  \item[a.] Modélisez cette situation comme un problème de graphes. Comment trouver le plus grand nombre de lettres non-confondables possible?
  \item[b.] On a besoin de $r$ symboles pour notre présentation. A partir de combien de couples de lettres confondables risque-t-on de devoir utiliser des symboles pouvant prêter à confusion?
\end{enumerate}


\paragraph{2. } Dans un groupe de 9 personnes, une personne connait 2 autres personnes, ces 2 personnes connaissent chacune 4 personnes, 4 personnes connaissent chacune 5 personnes et les 2 dernières personnes connaissent chacune 6 personnes. Montrez qu'il existe 3 personnes qui se connaissent l'une l'autre.


\paragraph{3. } Une fête regroupe $n$ personnes, chacune ayant au moins un ami présent (l'amitié étant réciproque). Dans tout groupe d'au moins 3 personnes, il n'y a jamais exactement 2 paires d'amis. Prouvez que chaque personne est amie avec toutes les autres.


\paragraph{4. } Montrez que le graphe $G$ est biparti si et seulement si $\alpha(H) \geq \frac{1}{2} \nu(H)$ pour tout sous-graphe $H$ de $G$.
\begin{itemize}
  \item $\alpha(H)$ est le nombre de sommets dans un ensemble indépendant maximum de $H$;
  \item $\nu(H)$ est le nombre de sommets de $H$.
\end{itemize}


\paragraph{5. } Sans utiliser le Théorème de Turán, montrez que si le graphe $G = (V, E)$ est simple et que $|E| > \frac{|V|^2}{4}$, alors $G$ contient un triangle.
